\section{Topic 3}

\subsection{Description Topic}
Si sviluppi un sistema informativo, composto da una base di dati relazionale e da un applicativo Java dotato
di GUI (Swing o JavaFX), per la gestione di calciatori di tutto il mondo.
Ogni calciatore è caratterizzato da nome, cognome, data di nascita, piede (sinistro, destro o ambidestro), uno
o più ruoli di gioco (portiere, difensore, centrocampista, attaccante) e una serie di feature caratteristiche (ad
esempio colpo di testa, tackle, rovesciata, etc.).
Il giocatore ha una carriera durante la quale può militare in diverse squadre di calcio. La militanza in una
squadra è caratterizzata da uno o più periodi di tempo nei quali il giocatore era in quella squadra. Ogni
periodo di tempo ha una data di inizio ed una data di fine. Durante la militanza del giocatore nella squadra si
tiene conto del numero di partite giocate, del numero di goal segnati e del numero di goal subiti (applicabile
solo ai giocatori di ruolo portiere). Il giocatore può inoltre vincere dei trofei, individuali o di squadra.
Il giocatore può avere anche una data di ritiro a seguito della quale decide di non giocare più. Le squadre di
calcio sono specificate dal loro nome e nazionalità.
L’amministratore del sistema si identifica con una login ed una password e ha il diritto di inserire nuovi
giocatori nella base di dati, modificarne i dati, aggiungere ulteriori informazioni oppure eliminare un
giocatore.
L’utente generico può vedere l’elenco dei giocatori e le loro caratteristiche e può richiedere diverse ricerche,
ad esempio filtrando i giocatori per nome, per ruolo, per piede, per numero di goal segnati, per numero di
goal subiti, per età, per squadre di appartenenza.
Per gruppi da 3 persone: I giocatori dopo la fine della carriera possono diventare allenatori o dirigenti. Il
sistema continua a mantenere una parte delle informazioni (squadra, numero di partite e trofei vinti) anche
per allenatori e dirigenti.
Inoltre, può accedere al sistema anche un terzo tipo di utente, consistente nel Giocatore stesso. Egli ha una
sua login e password e può modificare unicamente i dati relativi a sé stesso.
