\newpage
\subsection{\Large Analisi dei Requisiti}

Nell'affrontare questo progetto abbiamo deciso, su consiglio dei Docenti, di analizzare
la traccia proposta da una prospettiva piu' ampia; lasciando ai calciatori un ruolo centrale
ma inserendoli in un contesto piu' ampio e complesso che e' il mondo del calcio.

Per riuscire nel nostro intento abbiamo quindi dovuto svolgere numerose indagini e ricerche
sui principali siti di informazione riguardanti il calcio; nello specifico ci siamo basati sulle
informazioni reperite su Wikipidia, siti ufficiali delle principali confederazioni calcistiche,
Transfermarkt, Fantagazzetta, FootballManager e vari altri siti e giornali di stampo calcistico.

Con non poca difficolta' abbiamo cercato di astrarre quelle che sembravano essere le
caratteristiche fondamentali dei sitema calcio nel mondo e di inserirle, al meglio delle nostre
possibilita', nel progetto assegnatoci.

Una prima costante che abbiamo notato sin da subito e' che il sistema calcio e' un sistema
gerarchico strutturato come un vero e proprio governo e formato da confederazioni internazionali,
federazioni nazionali e leghe calcistiche (es. FIFA, UEFA, COMBEBOL, FIGC, ...) che definiscono
le regole del gioco, organizzano le varie competizioni e sono pertanto il fulcro di
tutto il sistema calcio.

NOTA. Per semplicita' a partire da questo momento con il termine confederazione faremo
riferimento indistintamente a confederazioni, federazioni, leghe o qualasi altro ente
organizzativo che abbia un ruolo nel governo del calcio.

Ogni confederazione e' associata con uno ed un solo paese.

NOTA. Come per le confederazioni anche il concetto di paese per noi sara' un'astrazione.
Identificheremo come paese i concetti di regione, nazione, continente e persino il mondo intero.

Data la relazione che intercorre tra un paese ed una confederazione calcistica, queste ultime,
potranno essere suddivise in tre grandi categorie:
\begin{itemize}
	\item confederazioni nazionali;
	\item confederazioni continentali;
	\item confederazioni mondiali.
\end{itemize}

Sottolineiamo che tale suddivisione e' in un certo senso una semplificazione in quanto
esistono confederazioni subcontinentali, confederazioni regionali e cosi' via; ci e' sembrato
superfuo entrare cosi' eccessivamente nello specifico visto il nostro scopo.

D'altra parte risulta immediato che il concetto di paese non sia legato esclusivamente a quello
di confederazione calcistica; (come anche sottolineato nella traccia) una squadra di calcio,
sia essa di tipo club o nazionale, chiaramente appartiene ad una nazione, e lo stesso discorso
si puo' estendere al concetto di calciatore, e ancor di piu' di persona.
Ogni calciatore avra' una nazione di nascita e, possibilmente, varie nazionalita'.

Il concetto di paese, quindi, risulta centrale per il mondo del calcio e pertanto per i nostri
scopi sara' necessario mapparlo come classe.

Ritornando al concetto di confederazione, una costante che abbiamo ritrovato grazie alle nostre
ricerche, e' che, come gia' accennato, sono organizzate in ordine gerarchico: le confederazioni
nazionali possono essere membro di confederazioni continentali che a loro volta possono essere
membro di confederazioni mondiali (es. la FIGC e' membro della UEFA che a sua volta e' membro
della FIFA).
Questo pattern si ripresenta per la quasi totalita' delle confederazioni che abbiamo osservato,
le uniche e rare eccezioni fanno capo a confederazioni minori isolate che quindi non sono in
associazione con nessun'altra confederazione.

E'indiscutibile che il concetto di confederazione calcistica, assieme a quello di paese,
formano la base sulla quale si poggia tutta l'organizzazione calcistica mondiale, e pertanto
nel nostro progetto avremo bisogno di una classe confederazione.

Come detto in precedenza, ciascuna confederazione organizza diverse competizioni calcistiche.
Una competizione calcistica puo'essere analizzata sotto diversi punti di vista;
se prendiamo come elemento discriminante il tipo di squadra che puo' parteciparvi, e' possibile
dividere le competizioni calcistiche in due grandi categorie:
\begin{itemize}
	\item competizioni per squadre di tipo club;
	\item competizioni per squadre di tipo nazionale.
\end{itemize}

Tuttavia, se come elemento discriminante prendiamo la formula che caratterizza una competizione
calcistica, queste possono essere raggruppate in tre grandi categorie
\begin{itemize}
	\item competizioni di tipo campionato;
	\item competizioni di tipo torneo;
	\item competizioni di tipo supercoppa.
\end{itemize}

Ad essere onesti, l'analisi delle competizioni calcistiche e' ben piu' complessa di quella
che stiamo mostrando in questo momento. Ogni competizione non solo ha una formula, che per altro
puo' cambiare in base all'edizione della competizione che si sta considerando, ma anche un numero
di partecipanti che e' variabile e pertanto risulta estremamente complicato astrarre tutte
queste informazioni.
Nonostante le difficolta' abbiamo tentato di mantenere entrambi i punti di vista sulle
competizioni calcistiche e, lasciandoci guidare dalla traccia, abbiamo scelto di preferire una
delle due prospettive.

Nella traccia si fa un chiaro riferimento alle squadre di calcio e pertanto ci e' sembrato
piu' corretto prediligere la prospettiva che mettesse in risalto questo aspetto maggiormente;
pertanto per i nostri scopi il concetto di competizione calcistica sara' in primo luogo
considerato in base al tipo di squadra che puo' parteciparvi e solo in secondo luogo
considerato in base al tipo di formula.

Risulta immediato che anche il concetto di competizione calcistica sara' mappato come classe.

