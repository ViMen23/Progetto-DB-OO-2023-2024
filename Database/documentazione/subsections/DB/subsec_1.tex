\subsection{\Large Progettazione Concettuale}

\newpage

\subsubsection{\Large Class Diagram non ristrutturato}

\includegraphics[width=\textwidth]{res/class_diagram_not_ristr}
\newpage

\subsubsection{\Large Ristrutturazione del Class Diagram}

\textbf{\large Analisi delle Ridondanze}

Possiamo constatare che non ci sono attributi ridondanti.

Notiamo però che vista l'alta frequenza di certe operazioni, 
che dovranno essere compiute sul database, vi è
un impellente necessità di introdurre delle ridondanze.


Prima operazione su tutte, è la visualizzazione di un 
resoconto delle statistiche di un calciatore rispetto alla 
sua militanza
(operazione richiesta anche dalla traccia).
Quest'ultimo infatti si potrebbe ottenere facendo una somma 
delle  statistiche riferite ai Play che sono associati allo 
specifico calciatore, alla squadra della militanza che si sta 
tenendo in considerazione e che siano avvenuti nell'arco di 
tempo in cui il calciatore in questione militava per quella 
certa squadra.


Per evitare che ad ogni richiesta di questa operazione si 
debba rifare il calcolo, soffermandoci nuovamente sul fatto 
che si suppone che questa operazione venga effettuata un 
numero molto elevato di volte, si vede la necessità di 
aggiungere in \textbf{Militanza} un attributo \textbf{match} 
che conservi il numero di presenze in campo di un calciatore 
durante quella specifica militanza.

Inoltre, vi è necessario aggiungere una classe 
\textbf{MilitancyStatistic} che sarà associata a 
\textbf{Statistic} e a \textbf{Militancy}, e che conterrà 
come attributo, \textbf{score} che ha lo scopo di conservare 
il valore di una certa statistica di un calciatore durante 
una specifica militanza.


Un'altra operazione che si suppone sia ricorrente, è il 
resoconto
del numero di presenze in campo di un calciatore in una 
determinata posizione.
Questo calcolo si potrebbe ottere dalla somma dell'attributo 
\textbf{match} delle tuple in Play associate al calciatore in 
quella determinata posizione.

Per velocizzare l'operzione si vede necessario, dunque, 
introdurre nella classe \textbf{PlayPosition} un attributo 
\textbf{match} che conservi il numero di presenze in campo di 
un calciatore in una specifica posizione.


L'ultima operazione, ovvero la visualizzazione delle 
nazionalità di un giocatore, si potrebbe ottenere navigando 
entrambe le associazioni che \textbf{Player} ha con 
\textbf{Country}: bornCountry e Nationality.

Visto che il nostro database ha come scopo primario l'essere 
un sistema informativo per i calciatori, si vede necessario 
aggiungere alla relazione Nationality anche la nazione di 
nascita del calciatore così da velocizzare l'operazione, in 
quanto così facendo basta navigare soltanto l'associazione 
Nationality.

\newpage
\textbf{\large Eliminazione delle Generalizzazioni}

Nonostante le numerose generalizzazioni presenti nel Class 
Diagram, tutte hanno un qualcosa in comune, ovvero sono 
generalizzazioni disgiunte totali, e in particolare le classi 
figlie non hanno attributi propri, ma al massimo associazioni 
con altre classi.

A seguito di ciò dunque, come metodo per eliminarle, si 
deciso di adottare l'accorpamento delle figlie della 
generalizzazione nel padre, aggiungendo dove necessario un 
attributo \textbf{type} che distinguesse le varie istanze 
della classe padre, e facendo le dovute correzioni ad 
eventuali associazioni delle classi figlie con le altre 
classi.

Di seguito viene mostrato un elenco delle generalizzazioni e 
di come sono state ristrutturate:

\bigskip
\textbf{Country}
\bigskip

Alla classe \textbf{Country} è stato aggiunto un attributo 
\textbf{type} di tipo enum chiamato \textbf{enCountry}.
Il tipo \textbf{enCountry} può assumere tre valori:
\begin{itemize}
	\item Nation;
	\item Continental;
	\item World.
\end{itemize}

Per quanto riguarda le associazioni:

Le associazioni delle classi figlie di \textbf{Country} con 
le classi figlie di \textbf{Confederation} sono state 
accorpate, risultandone in un unica associazione tra 
\textbf{Country} e \textbf{Confederation} con molteplicità 
invariata.

Per quanto riguarda le associazioni della classe figlia 
\textbf{Nation} con \textbf{Player} e \textbf{Team}, esse 
sono state ricollegate alla classe padre \textbf{Country} 
mantenendo la molteplicità invariata.

\bigskip
\textbf{Confederation}
\bigskip

Alla classe \textbf{Confederation}  non è stato aggiunto un 
attributo \textbf{type} poiché si è sfruttato l'associazione 
1 a 1 che le classi figlie di \textbf{Confederation} avevano 
con le classi figlie di \textbf{Country}.
Dunque per comprendere a quale classe figlia una tupla di 
\textbf{Confederation} corrisponde, bisogna basarsi 
sull'attributo \textbf{type} della classe \textbf{Country}.

Per quanto riguarda le associazioni:

Le due associazioni che intercorrono tra le classi figlie di 
\textbf{Confederation}, ovvero l'associazione tra 
\textbf{NationalConfederation} e 
\textbf{ContinentalConfederation}, e tra 
\textbf{ContinentalConfederation} e 
\textbf{WorldConfederation}, sono state accorpate, 
risultandone in un'unica associazione tra la classe 
\textbf{Confederation} e sé stessa.

Le due associazioni delle classi figlie 
\textbf{ContinentalConfederation} e 
\textbf{WorldConfederation} con \textbf{NationalCompetition}, 
sono state accorpate, risultandone in un'unica associazione 
tra la classe \textbf{Confederation} e la classe 
\textbf{Competition}, classe padre di 
\textbf{NationalCompetition}.

\newpage
\textbf{Competition}
\bigskip

Alla classe \textbf{Competition} sono stati aggiunti due 
attributi: \textbf{type} di tipo enum chiamato 
\textbf{enCompetition} e \textbf{teamType} di tipo enum 
chiamato \textbf{enTeam}. Questo perché anche le classi 
figlie di \textbf{Competition} erano una generalizzazione a 
loro volta.

Nota che ci si è ridotti a solo due tipi aggiunti, perché le 
classi \textbf{NationalCompetition} e {ClubCompetition} hanno 
le stesse classi figlie.

Il tipo enum \textbf{enTeam} può assumere i seguenti valori:
\begin{itemize}
	\item National;
	\item Club.
\end{itemize}

Il tipo enum \textbf{enCompetition} può assumere i seguenti 
valori:
\begin{itemize}
	\item Cup;
	\item League;
	\item Super Cup.
\end{itemize}

Per quanto riguarda le associazioni:

L'associazione tra \textbf{ClubCompetition} e 
\textbf{Confederation}, è stata ricollegata alla classe padre 
\textbf{Competition} mantenendone però la molteplicità 
invariata.


Le associazioni tra \textbf{NationalCompetition} e  
\textbf{NationalCompetitionEdition} e tra 
\textbf{ClubCompetition} e \textbf{ClubCompetitionEdition} 
sono state accorpate, risultandone in un'unica associazione 
(con molteplicità invariata) tra \textbf{Competition} e 
\textbf{CompetitionEdition}, la classe padre di 
\textbf{NationalCompetitionEdition} e 
\textbf{ClubCompetitionEdition}.

\bigskip
\textbf{CompetitionEdition}
\bigskip

Alla classe \textbf{CompetitionEdition} non è stato aggiunto 
un attributo \textbf{type} poiché si è sfruttata 
l'associazione che vi era tra le sue classi figlie e le 
classi figlie di \textbf{Competition}. Dunque per comprendere 
a quale classe figlia una tupla di 
\textbf{CompetitionEdition} corrisponde, bisogna basarsi 
sull'attributo \textbf{teamType} della classe 
\textbf{Competition}.

Per quando riguarda le associazioni:

Le associazioni tra \textbf{NationalCompetitionEdition} e 
\textbf{NationalPartecipation} e tra 
\textbf{ClubCompetitionEditon} e \textbf{ClubPartecipation}, 
sono state accorpate, risultandone in un'unica associazione 
(con molteplicità invariata) tra la classe 
\textbf{CompetitionEdition} e \textbf{Partecipation}, la 
classe padre di \textbf{NationalPartecipation} e 
\textbf{ClubPartecipation}.


\bigskip
\textbf{Team}
\bigskip

Alla classe \textbf{Team} è stato aggiunto un attributo 
\textbf{type} di tipo enum chiamato \textbf{enTeam} (lo 
stesso di Competition).

Per quanto riguarda le associazioni:

Le associazioni tra \textbf{National} e 
\textbf{NationalPartecipation} e tra \textbf{Club} e 
\textbf{ClubPartecipation}, sono state accorpate, 
risultandone in un'unica associazione tra la classe 
\textbf{Team} e \textbf{Partecipation} la classe padre di
\textbf{NationalPartecipation} e \textbf{ClubPartecipation}


\newpage

\textbf{Partecipation}
\bigskip

Alla classe \textbf{Partecipation} non è stato aggiunto un 
attributo \textbf{type} poiché si è sfruttata l'associazione 
che vi era tra le sue classi figlie e le classi figlie di 
\textbf{Team}. Dunque per comprendere a quale classe figlia 
una tupla di \textbf{Partecipation} corrisponde, bisogna 
basarsi sull'attributo \textbf{type} della classe 
\textbf{Team}.

Per quanto riguarda le associazioni:
Sono già state gestite precedentemente.

\bigskip
\textbf{Trophy}
\bigskip

Alla classe \textbf{Trophy} è stato aggiunto un attributo 
\textbf{type} di tipo enum chiamato \textbf{enAward}.

Il tipo \textbf{enAward} assume i seguenti valori:
\begin{itemize}
	\item Player
	\item Team
\end{itemize}


Per quanto riguarda le associazioni:

L'associazione tra \textbf{TeamTrophy} e 
\textbf{Partecipation} è stata ricollegata alla classe padre 
\textbf{Trophy} mantenendo la molteplicità invariata.


\bigskip
\textbf{Prize}
\bigskip

Alla classe \textbf{Prize} è stato aggiunto un attributo 
\textbf{type} di tipo enum chiamato \textbf{enAward} (lo 
stesso tipo di Trophy).

Per quanto riguarda le associazioni:

L'associazione tra \textbf{TeamPrize} e 
\textbf{TeamPrizeCase} è stata ricollegata alla classe padre 
\textbf{Prize} mantenendo la molteplicità invariata.


L'associazione tra \textbf{PlayerPrize} e 
\textbf{PlayerPrizeCase} è stata ricollegata alla classe 
padre \textbf{Prize} mantenendo la molteplicità invariata.

\bigskip
\textbf{\large Eliminazioni Attributi Multivalore}
\bigskip

Vi sono tre attributi multivalore presenti nel Class Diagram:
\begin{itemize}
	\item L'attributo \textbf{role}
		della classe \textbf{Prize};
	\item L'attributo \textbf{role}
		della classe \textbf{Trophy};
	\item L'attributo \textbf{role}
		della classe \textbf{Statistic}.
\end{itemize}

Questo attributo assume la stessa funzione in tutte e tre le 
Classi, ovvero quello di descrivere quella determinata classe 
a quali ruoli può essere associata.


Data l'importanza assunta da quest'attributo, si è deciso di 
creare un tipo enum che contiene tutte le combinazioni di 
ruoli possibili (ci possono essere massimo 4 ruoli).

\newpage
\textbf{\large Eliminazione Attributi Strutturati}
\bigskip

Nel Class Diagram non sono presenti Attributi Strutturati.

\bigskip
\textbf{\large Partizionamento/Accorpamento di Entità
		e Associazioni}
\bigskip

L'unica accorpamento che potrebbe essere effettuato nel Class 
Diagram è quello tra \textbf{Country} e 
\textbf{Confederation}, poiché tra esse vi è un'associazione 
1 a 1.

Si è arrivati però alla conclusione che c'è bisogno che le 
due classi siano separate, perché nelle altre associazioni 
che esse hanno con le altre classi, svolgono un ruolo 
centrale.

\bigskip
\textbf{\large Scelta degli Identificatori Primari}
\bigskip

\newpage
\subsubsection{\Large Class Diagram ristrutturato}
\includegraphics[width=\textwidth]{res/class_diagram_ristr}
\newpage

\subsubsection{\Large Dizionario}

\begin{center}
	\textbf{Dizionario delle Classi}
\end{center}


\begin{tblr}{
    hlines = {0.9pt}, vlines = {0.9pt}, colspec = {X[l]X[l]X[l]}, column{1}= {100pt},
    width = \textwidth, cell{1}{1-3} = {blue!10!white}
}
	{
		Classe
	}
	&
	{
		Descrizione
	}
	&
	{
		Attributo
	}
	\\
	{
		Attribute
	}
	&
	{
		Rappresenta gli attributi di un calciatore.
	}
	&
	{
		\textbf{id}(Integer)[chiave surrogata]:\\Rappresenta
			l'identificativo di un Attributo.\\
		\medskip\textbf{type}(enFeature):\\Rappresenta
			il tipo di un Attributo.\\
		\medskip\textbf{name}(String)[chiave naturale]:
			\\Rappresenta il nome di un Attributo.\\
		\medskip\textbf{description}(String)[parziale]:
			\\Rappresenta la descrizione di un Attributo.
	}
	\\
	{
		Competition
	}
	&
	{
		Rappresenta le competizioni calcistiche.
	}
	&
	{
		\textbf{id}(Integer)[chiave surrogata]:\\Rappresenta
			l'identificativo di una Competizione.\\
		\medskip\textbf{type}(enCompetition):\\Rappresenta
			il tipo di una Competizione.\\
		\medskip\textbf{teamType}(enTeam):\\Rappresenta
			il tipo di squadra che può
			partecipare alla Competizione.\\
		\medskip\textbf{name}(String)[chiave naturale]:
			\\Rappresenta il nome di una Competizione.\\
		\medskip\textbf{frequency}(Integer):\\Rappresenta
			la frequenza di una Competizione.
	}
	\\
	{
		CompetitionEdition
	}
	&
	{
		Rappresenta le edizioni delle competizioni calcistiche.
	}
	&
	{
		\textbf{startYear}(Integer)[chiave parziale]:
			\\Rappresenta l'anno di inizio di un'Edizione.\\
		\medskip\textbf{endYear}(Integer)[chiave parziale]:
			\\Rappresenta l'anno di fine di un'Edizione.\\
		\medskip\textbf{totalTeam}(Integer):\\Rappresenta
			il numero di team che partecipano in un'Edizione.
	}
	\\
	{
		Confederation
	}
	&
	{
	Rappresenta le confederazioni calcistiche.
	}
	& 
	{
		\textbf{id}(Integer)[chiave surrogata]:\\Rappresenta
			l'identificativo di una Confederazione.\\
		\medskip\textbf{shortName}(String):\\Rappresenta
			il nome abbreviato di una Confederazione.\\
		\medskip\textbf{longName}(String)[chiave naturale]:
			\\Rappresenta il nome esteso di una Confederazione.
	}
	\\
\end{tblr}

\newpage

\begin{tblr}{
    hlines = {0.9pt}, vlines = {0.9pt}, colspec = {X[l]X[l]X[l]}, column{1}= {100pt},
    width = \textwidth
}
	{
		Country
	}
	&
	{
		Rappresenta i paesi in cui si gioca
		ufficialmente a calcio.
	}
	&
	{
		\textbf{id}(Integer)[chiave surrogata]:\\Rappresenta
			l'identificativo di un Paese.\\
		\medskip\textbf{type}(enCountry):\\Rappresenta
			il tipo di un Paese.\\
		\medskip\textbf{code}(String)[chiave naturale]:
			\\Rappresenta il codice ISO 3166-1 alpha-3
			di un Paese.\\
		\medskip\textbf{name}(String)[chiave naturale]:
			\\Rappresenta il nome di un Paese.
	}
	\\
	{
		Player
	}
	&
	{
		Rappresenta i calciatori.
	}
	&
	{
		\textbf{id}(Integer)[chiave surrogata]:\\Rappresenta
			l'identificativo di un Calciatore.\\
		\medskip\textbf{name}(String):\\Rappresenta
			il nome di un Calciatore.\\
		\medskip\textbf{surname}(String):\\Rappresenta
			il cognome di un Calciatore.\\
		\medskip\textbf{dob}(Date):\\Rappresenta
			la data di nascita.\\
		\medskip\textbf{foot}(enFoot):\\Rappresenta
			il piede preferito di un Calciatore.\\
		\medskip\textbf{role}(enRoleMix)[derivato, parziale]:
			\\Rappresenta i possibili ruoli di gioco
			di un Calciatore.
	}
	\\
	{
		PlayerRetired
	}
	&
	{
		Rappresenta i calciatori che sono ritirati.
	}
	&
	{
		\textbf{retiredDate}(Date):\\Rappresenta
			la data di ritiro di un calciatore.
	}
	\\
	{
		Position
	}
	&
	{
		Rappresenta le posizioni di gioco di un Calciatore.
	}
	&
	{
		\textbf{id}(Integer)[chiave surrogata]:\\Rappresenta
			l'identificativo di una Posizione.\\
		\medskip\textbf{role}(enRole):\\Rappresenta
			il ruolo associato ad una Posizione.\\
		\medskip\textbf{code}(String)[chiave naturale]:
			\\Rappresenta il nome abbreviato di una Posizione.\\
		\medskip\textbf{name}(String)[chiave naturale]:
			\\Rappresenta il nome di una Posizione.
	}
	\\
\end{tblr}

\newpage

\begin{tblr}{
    hlines = {0.9pt}, vlines = {0.9pt}, colspec = {X[l]X[l]X[l]}, column{1}= {100pt},
    width = \textwidth
}

	{
		Prize
	}
	&
	{
		Rappresenta i premi calcistici.
	}
	&
	{
		\textbf{id}(Integer)[chiave surrogata]:\\Rappresenta
			l'identificativo del Premio.\\
		\medskip\textbf{type}(enAward):\\Rappresenta
			il tipo del Premio.\\
		\medskip\textbf{role}(enRole)[parziale]:\\Rappresenta
			il ruolo a cui è associato un Premio.\\
		\medskip\textbf{name}(String)[chiave naturale]:
			\\Rappresenta il nome del Premio.\\
		\medskip\textbf{description}(String)[parziale]:
			\\Rappresenta la descrizione del Premio.\\
		\medskip\textbf{given}(String):\\Rappresenta
			il nome della società calcistica
			che conferisce il Premio.
	}
	\\
	{
		Statistic
	}
	&
	{
		Rappresenta le statistiche di un calciatore.
	}
	&
	{
		\textbf{id}(Integer)[chiave surrogata]:\\Rappresenta
			l'identificativo della Statistica.\\
		\medskip\textbf{goalkeeper}(Boolean):\\Rappresenta
			se true che la Stastistica è associata
			soltanto al portiere, altrimenti a tutti i ruoli.\\
		\medskip\textbf{name}(String):\\Rappresenta
			il nome della Statistica.\\
		\medskip\textbf{description}(String)[parziale]:
			\\Rappresenta la descrizione della Statistica.
	}
	\\
	{
		Tag
	}
	&
	{
		Rappresenta i tag di un calciatore.
	}
	&
	{
		\textbf{id}(Integer)[chiave surrogata]:\\Rappresenta
			l'identificativo del Tag.\\
		\medskip\textbf{type}(enFeature):\\Rappresenta
			il tipo del Tag.\\
		\medskip\textbf{name}(String)[chiave naturale]:
			\\Rappresenta il nome del Tag.\\
		\medskip\textbf{description}(String)[parziale]:
			\\Rappresenta la descrizione del Tag.
	}
	\\
	{
		Team
	}
	&
	{
		Rappresenta le squadre di calcio.
	}
	&
	{
		\textbf{id}(Integer)[chiave surrogata]:\\Rappresenta
			l'identificativo della Squadra.\\
		\medskip\textbf{type}(enTeam):\\Rappresenta
			il tipo della Squadra.\\
		\medskip\textbf{name}(String)[chiave naturale]:
			\\Rappresenta il nome della Squadra.
	}
	\\
\end{tblr}

\newpage

\begin{tblr}{
    hlines = {0.9pt}, vlines = {0.9pt}, colspec = {X[l]X[l]X[l]}, column{1}= {100pt},
    width = \textwidth
}

	{
		Trophy
	}
	&
	{
		Rappresenta i trofei calcistici.
	}
	&
	{
		\textbf{id}(Integer)[chiave surrogata]:\\Rappresenta
			l'identificativo del Trofeo.\\
		\medskip\textbf{type}(enAward):\\Rappresenta
			il tipo del Trofeo.\\
		\medskip\textbf{role}(enRole)[parziale]:\\Rappresenta
			il ruolo a cui è associato un Trofeo.\\
		\medskip\textbf{name}(String)[chiave naturale]:
			\\Rappresenta il nome del Trofeo.\\
		\medskip\textbf{description}(String)[parziale]:
			\\Rappresenta la descrizione del Trofeo.\\
	}
	\\
	{
		UserAccount
	}
	&
	{
		Rappresenta gli utenti dell'applicativo.
	}
	&
	{
		\textbf{username}(String)[chiave naturale]:\\Rappresenta
			l'username dell'Account dell'Utente.\\
		\medskip\textbf{password}(String):\\Rappresenta
			la password dell'Account dell'Utente.\\
		\medskip\textbf{priviledge}(Integer):\\Rappresenta
			i privilegi dell'Account dell'Utente.
	}
	\\
\end{tblr}

\newpage

\begin{center}
	\textbf{Dizionario delle Associazioni}
\end{center}


\begin{tblr}{
    hlines = {0.9pt}, vlines = {0.9pt}, colspec = {X[l]X[l]X[l]X[l]}, column{1-2}= {100pt},
    width = \textwidth, cell{1}{1-4} = {blue!10!white}
}

	{
		Nome
	}
	&
	{
		Descrizione
	}
	&
	{
		Classe in Relazione
	}
	&
	{
		Attributo
	}
	\\
	{
		\textbf{Militancy}
	}
	&
	{
		Esprime le militanze di un calciatore in una squadra.
	}
	&
	{
		\textbf{Player [0 ... *]}:\\Indica che un Calciatore
			può militare in più Squadre.\\
		\medskip\textbf{Team [0 ... *]}:\\Indica che una Squadra
			può essere associata a più Calciatori.
	}
	&
	{
		\textbf{id}(Integer)[chiave surrogata]:\\Rappresenta
			l'identificativo di una Militanza.\\
		\medskip\textbf{startSeason}(Integer):\\Rappresenta
			la stagione d'inizio di una Militanza.\\
		\medskip\textbf{typeStartSeason}(enSeason):\\Rappresenta
			in che parte della stagione è iniziata la Militanza.\\
		\medskip\textbf{endSeason}(Integer):\\Rappresenta
			la stagione in cui termina la Militanza.\\
		\medskip\textbf{typeEndSeason}(enSeason):\\Rappresenta
			in che parte della stagione è terminata la Militanza.
	}
	\\
	{
		\textbf{PlayerPrizeCase}
	}
	&
	{
		Esprime i premi di un calciatore.
	}
	&
	{
		\textbf{Player [0 ... *]}:\\Indica che
			un Calciatore può essere associato a più Premi.\\
		\medskip\textbf{Prize [0 ... *]}:\\Indica che
			un Premio può essere associato, in anni diversi,
			a più Premi.
				
	}
	&
	{
		\textbf{assignYear}(Integer):\\Rappresenta
			l'anno di assegnazione del Premio al Calciatore.
	}
	\\
	{
		\textbf{Nationality}
	}
	&
	{
		Esprime le nazionalità di un calciatore.
	}
	&
	{
		\textbf{Country [0 ... *]}:\\Indica che
			uno stesso paese può essere associato a più
			calciatore.\\
		\medskip\textbf{Player [0 ... *]}:\\Indica che
			un calciatore può avere più nazionalità.
	}
	&
	{
	
	}
	\\
	{
		\textbf{bornCountry}
	}
	&
	{
		Esprime il paese di nascita di un calciatore.
	}
	&
	{
		\textbf{Country [0 ... *]}:\\Indica che
			un paese può essere il paese di nascita
			di più calciatori.\\
		\medskip\textbf{Player [1]}:\\Indica che
			un calciatore ha uno e un solo paese di nascita.
	}
	&
	{
	
	}
	\\
	{
		\textbf{Player-PlayerRetired}
	}
	&
	{
		Esprime i calciatori ritirati.
	}
	&
	{
		\textbf{Player [0 ... 1]}:\\Indica che
			un Calciatore può essere o non essere ritirato.\\
		\medskip\textbf{PlayerRetired [1]}:\\Indica che
			una data di ritiro si riferisce ad uno
			e un solo Calciatore. 
	}
	&
	{
	
	}
	\\
\end{tblr}

\newpage

\begin{tblr}{
    hlines = {0.9pt}, vlines = {0.9pt}, colspec = {X[l]X[l]X[l]X[l]}, column{1-2}= {100pt},
    width = \textwidth
}

	{
		\textbf{Player-Tag}
	}
	&
	{
		Esprime i tag di un calciatore.
	}
	&
	{
		\textbf{Player [0 ... *]}:\\Indica che
			un Calciatore può essere associato a più Tag.\\
		\medskip\textbf{Tag [0 ... *]}:\\Indica che
			uno Tag può essere associato a più Calciatori.
	}
	&
	{
		
	}
	\\
	{
		\textbf{Player-Position}
	}
	&
	{
		Esprime le posizioni di gioco di un calciatore.
	}
	&
	{
		\textbf{Player [0 ... *]}:\\Indica che
			un Calciatore può essere associato a più Posizioni.\\
		\medskip\textbf{Position [0 ... *]}:\\Indica che
			una Posizione può essere associata a più Calciatori.
	}
	&
	{
		
	}
	\\
	{
		\textbf{PlayerAttribute}
	}
	&
	{
		Esprime gli attributi di un calciatore.
	}
	&
	{
		\textbf{Player [0 ... *]}:\\Indica che un Calciatore
			può essere associato a più Attributi.\\
		\medskip\textbf{Attribute [0 ... *]}:\\Indica che 
			un Attributo può essere associato a più Calciatori.
	}
	&
	{
		\textbf{score}(Integer):\\Rappresenta il valore
			di un Attributo associato al Calciatore.
	}
	\\
	{
		\textbf{Squad}
	}
	&
	{
		Esprime la rosa in un determinato anno di una squadra.
	}
	&
	{
		\textbf{Militancy [0 ... *]}:\\Indica che
			una Militanza può essere associato a più Rose.\\
		\medskip\textbf{Player [0 ... *]}:\\Indica che
			un Calciatore può essere associato a più Rose.\\
		\medskip\textbf{Team [0 ... *]}:\\Indica che
			una Squadra può essere associata a più Rose.\\
		\medskip\textbf{Squad -> Team [1]}:\\Indica che
			una Rosa può essere associata ad una
			e una sola Squadra.\\
		\medskip\textbf{Squad -> Militancy [1]}:\\Indica che
			una Rosa può essere associata ad una
			e una sola Militanza.\\
		\medskip\textbf{Squad -> Player [1]}:\\Indica che
			una Rosa può essere associata ad un
			e un solo Calciatore.
	}
	&
	{
		\textbf{startYear}(Integer):\\Rappresenta
			la stagione di riferimento di una Rosa.\\
		\medskip\textbf{type}(enSeason):\\Rappresenta
			fino a che parte della stagione un Giocatore
			era in Rosa.
	}
	\\
	{
		\textbf{TeamPrizeCase}
	}
	&
	{
		Esprime i premi di una squadra.
	}
	&
	{
		\textbf{Team [0 ... *]}:\\Indica che
			una Squadra può essere associata a più Premi.\\
		\medskip\textbf{Prize [0 ... *]}:\\Indica che
			un Premio, in anni diversi, a più Squadre.
	}
	&
	{
		\textbf{assignYear}(Integer):\\Rappresenta
			l'anno di assegnazione di un Premio ad una Squadra.\\
	}
	\\
\end{tblr}

\newpage

\begin{tblr}{
    hlines = {0.9pt}, vlines = {0.9pt}, colspec = {X[l]X[l]X[l]X[l]}, column{1-2}= {100pt}, column{4}= {10pt},
    width = \textwidth
}

	{
		\textbf{Team-Country}
	}
	&
	{
		Esprime la nazionalità di una squadra.
	}
	&
	{
		\textbf{Team [1]}:\\Indica che una Squadra
			può essere associata ad un e un solo Paese.\\
		\medskip\textbf{Country [0 ... *]}:\\Indica che
			un Paese può essere associato a più Squadre.	
	}
	&
	{
		
	}
	\\
	{
		\textbf{Team-Confederation}
	}
	&
	{
		Esprime la confederazione di cui è membro una squadra.
	}
	&
	{
		\textbf{Team [1]}:\\Indica che una Squadra
			può essere associata ad un e un solo Paese.\\
		\medskip\textbf{Confederation [0 ... *]}:\\Indica che
			una Confederazione può essere associata
			a più Squadre.
	}
	&
	{
		
	}
	\\
	{
		\textbf{Partecipation}
	}
	&
	{
		Esprime  la partecipazione di una squadra
		ad un'edizione di una competizione.
	}
	&
	{
		\textbf{Team [0 ... *]}:\\Indica che una Squadra
			può partecipare a più Edizioni.\\
		\medskip\textbf{CompetitionEdition [0 ... *]}:
			\\Indica che una Edizione può essere associata
			a più squadre.

	}
	&
	{
		
	}
	\\
	{
		\textbf{Competition-CompetitionEdition}
	}
	&
	{
		Esprime le edizioni di una competizione.\\
		È una relazione identificante.
	}
	&
	{
		\textbf{Competition[0 ... *]}:\\Indica che
			una Competizione può essere associata
			a più Edizioni.\\
		\medskip\textbf{CompetitionEdition[1]}:\\Indica che
			un'Edizione può essere associata ad una
			e una sola Competizione.
	}
	&
	{
		
	}
	\\
	{
		\textbf{Competition-Confederation}
	}
	&
	{
		Esprime la confederazione che organizza
		la competizione.
	}
	&
	{
		\textbf{Competition [1]}:\\Indica che
			una Competizione può essere associata ad una
			e una sola Confederazione.\\
		\medskip\textbf{Confederation [0 ... *]}:\\Indica che
			una Confederazione può essere associata
			a più Competizioni.	
	}
	&
	{
		
	}
	\\
	{
		\textbf{Confederation-Confederation}
	}
	&
	{
		Esprime la possibilità di una confederazione di
		avere come membri altre confederazioni, o essere
		membro a sua volta.
	}
	&
	{
		\textbf{Confederation [0 ... 1] ruolo (contenuto)}:\\
			Indica che una Confederazione può o non può essere
			membra di un'altra Confederazione.\\
		\medskip\textbf{Confederation [0 ... *]
			ruolo (contiene)}:\\
			Indica che una Confederazione può essere associata
			a più Confederazioni.
	}
	&
	{
		
	}
	\\
	{
		\textbf{Confederation-Country}
	}
	&
	{
		Esprime l'appartenza di una confederazione
		ad un unico paese.
	}
	&
	{
		\textbf{Confederation [1]}:\\Indica che
			una Confederazione può essere associata
			ad un e un solo Paese.\\
		\medskip\textbf{Country [0 ... 1]}:\\Indica che
			un Paese può essere associata a nessuna o ad una
			Confederazione.
	}
	&
	{
		
	}
	\\
	{
		\textbf{Partecipation-Trophy}
	}
	&
	{
		Esprime la bacheca dei trofei di una squadra.
	}
	&
	{
		\textbf{Trophy [0 ... *]}:\\Indica che un Trofeo
			può essere associata a più Partecipazioni
			di una Squadra ad una Edizione.\\
		\medskip\textbf{Partecipation [0 ... *]}:\\Indica che
			una Partecipazione può essere associata
			a più Trofei.
	}
	&
	{
		
	}
	\\
\end{tblr}

\newpage

\begin{tblr}{
    hlines = {0.9pt}, vlines = {0.9pt}, colspec = {X[l]X[l]X[l]X[l]}, column{1-2}= {100pt},
    width = \textwidth
}

	{
		\textbf{PlayTrophyCase}	
	}
	&
	{
		Esprime la bacheca dei trofei di un calciatore.
	}
	&
	{
		\textbf{Partecipation [0 ... *]}:\\Indica che
			una Partecipazione può essere associata
			a più Bacheche.\\
		\medskip\textbf{Trophy [0 ... *]}:\\Indica che
			un Trofeo può essere associato
			a più Bacheche.\\
		\medskip\textbf{Squad [0 ... *]}:\\Indica che
			una Rosa può essere associata
			a più Bacheche.\\
		\medskip\textbf{PlayTrophyCase -> Partecipation [1]}:
			\\Indica che una Bacheca può essere associata
			ad una e una sola Partecipazione.\\
		\medskip\textbf{PlayTrophyCase -> Trophy [1]}:
			\\Indica che una Bacheca può essere associata
			ad un e un solo Trofeo.\\
		\medskip\textbf{PlayTrophyCase -> Squad [1]}:
			\\Indica che una Bacheca può essere associata
			ad una e una sola Rosa.
	}
	&
	{
		
	}
	\\
	{
		\textbf{Play}
	}
	&
	{
		Esprime in quale edizione un calciatore, in una squadra
		e in una certa stagione, ha giocato.
	}
	&
	{
		\textbf{Partecipation [0 ... *]}:\\Indica che
			una Partecipazione può essere associata
			a più Rose.\\
		\medskip\textbf{Squad [0 ... *]}:\\Indica che
			una Rosa può essere associata
			a più Partecipazioni.
	}
	&
	{
		\textbf{id}(Integer)[chiave surrogata]:\\Rappresenta
			l'identificativo di un Gioco.\\
		\medskip\textbf{match}(Integer):\\Rappresenta
			il numero di presenze di un calciatore
			in un Gioco.
	}
	\\
	{
		\textbf{PlayStatistic}
	}
	&
	{
		Esprime le statistiche di un gioco.
	}
	&
	{
		\textbf{Play [0 ... *]}:\\Indica che
			un Gioco può essere associato
			a più Statistiche.\\
		\medskip\textbf{Statistic [0 ... *]}:\\Indica che
			una Statistica può essere associata
			a più Giochi.\\
	}
	&
	{
		\textbf{score}(Integer):\\Rappresenta
			il valore di una Statistica per un Gioco.
	}
	\\
\end{tblr}


\newpage

\begin{center}
	\textbf{Dizionario dei Vincoli}
\end{center}

\begin{tblr}{
    hlines = {0.9pt}, vlines = {0.9pt}, colspec = {X[l]X[l]X[l]}, 
    width = \textwidth, cell{1}{1-3} = {blue!10!white}
}
	{
		Vincolo
	}
	&
	{
		Tipo
	}
	&
	{
		Descrizione
	}
	\\
	{
		ckCompetitionEditionRange
	}
	&
	{
		N-PLA
	}
	&
	{
		Ogni edizione di una competizione calcistica deve
		iniziare e finire nello stesso anno o
		al massimo terminare l'anno successivo
		a quello di inizio.
	}
	\\
	{
		ckCompetitionEditionTotalTeam
	}
	&
	{
		N-PLA
	}
	&
	{
		Il numero di squadre di calcio che possono partecipare
		ad una edizione di una competizione calcistica
		deve essere compreso tra un minimo di 2 ed
		un massimo di 128.
	}
	\\
	{
		ckMilitancyType
	}
	&
	{
		N-PLA
	}
	&
	{
		Tutti i tipi stagione (inizio o fine) in una Militanza
		possono essere soltanto prima parte o seconda parte.
	}
	\\
	{
		ckMilitancySeason
	}
	&
	{
		N-PLA
	}
	&
	{
		In una Militanza, ogni stagione d'inizio deve
		precedere o al massimo essere uguale
		alla stagione di fine.
	}
	\\
	{
		ckMilitancyValidCombo
	}
	&
	{
		N-PLA
	}
	&
	{
		Ogni militanza di un calciatore che inizi e termini
		nella stessa stagione, non può iniziare
		nella seconda parte e terminare nella prima parte.
	}
	\\
	{
		ckTrophy
	}
	&
	{
		N-PLA
	}
	&
	{
		I trofei di squadra non devono essere associati
		ad alcun ruolo.
	}
	\\
	{
		ckPrize
	}
	&
	{
		N-PLA
	}
	&
	{
		I premi di squadra non devono essere associati
		ad alcun ruolo.
	}
	\\
	{
		ckPlay
	}
	&
	{
		N-PLA
	}
	&
	{
		Il numero di presenze di un calciatore in un gioco
		deve essere maggiore di zero.
	}
	\\
	{
		
	}
	&
	{
		
	}
	&
	{
		
	}
	\\
	{
		
	}
	&
	{
		
	}
	&
	{
		
	}
	\\
	{
		
	}
	&
	{
		
	}
	&
	{
		
	}
	\\
	{
		
	}
	&
	{
		
	}
	&
	{
		
	}
	\\
	{
		
	}
	&
	{
		
	}
	&
	{
		
	}
	\\
	{
		
	}
	&
	{
		
	}
	&
	{
		
	}
	\\
	{
		
	}
	&
	{
		
	}
	&
	{
		
	}
	\\
	{
		
	}
	&
	{
		
	}
	&
	{
		
	}
	\\
	{
		
	}
	&
	{
		
	}
	&
	{
		
	}
	\\
	{
		
	}
	&
	{
		
	}
	&
	{
		
	}
	\\
	{
		
	}
	&
	{
		
	}
	&
	{
		
	}
	\\
	{
		
	}
	&
	{
		
	}
	&
	{
		
	}
	\\
	{
		
	}
	&
	{
		
	}
	&
	{
		
	}
	\\
	{
		
	}
	&
	{
		
	}
	&
	{
		
	}
	\\
	{
		
	}
	&
	{
		
	}
	&
	{
		
	}
	\\
	{
		
	}
	&
	{
		
	}
	&
	{
		
	}
	\\
	{
		
	}
	&
	{
		
	}
	&
	{
		
	}
	\\
	{
		
	}
	&
	{
		
	}
	&
	{
		
	}
	\\
	{
		
	}
	&
	{
		
	}
	&
	{
		
	}
	\\
	{
		
	}
	&
	{
		
	}
	&
	{
		
	}
	\\
	{
		
	}
	&
	{
		
	}
	&
	{
		
	}
	\\
	{
		
	}
	&
	{
		
	}
	&
	{
		
	}
	\\
\end{tblr}
