\subsection{Analisi dei Requisiti}

Nell'affrontare questo progetto abbiamo Nell'affrontare questo progetto abbiamo deciso - seguendo i suggerimenti dei Docenti -, di svolgere
la traccia proposta entro una cornice più ampia, preservando il ruolo centrale delle entità-calciatori,
ma ponendole al centro di un contesto più complesso, rappresentativo dell'intero mondo calcistico.

A questo scopo, abbiamo effettuato una serie di indagini e ricerche sui principali siti di informazione riguardanti il calcio. Segnatamente, ci siamo basati sulle informazioni reperibili da Wikipedia, siti ufficiali delle principali confederazioni calcistiche, Transfermarkt, Fantagazzetta, FootballManager e vari altri siti e giornali di argomento calcistico.

Con non poca difficoltà abbiamo cercato di astrarre quelle che, a una prima analisi, abbiamo giudicato essere le
caratteristiche fondamentali del sistema-calcio nel mondo reale e di configurarle, al meglio delle nostre
possibilità, nel progetto assegnatoci.

\bigskip
\bigskip

Un primo aspetto concettuale - che sin da subito è risultato evidente - è che il sistema-calcio è
gerarchico, strutturato come un vero e proprio governo e costituito da confederazioni internazionali,
federazioni nazionali e leghe calcistiche (es. FIFA, UEFA, COMBEBOL, FIGC, \dots), che definiscono
le regole del gioco, organizzano le varie competizioni e sono pertanto il fulcro dell'intero sistema.

Per semplicità, a partire da questo momento con il termine "confederazione" faremo
riferimento indistintamente a confederazioni, federazioni, leghe e a qualsiasi altro ente
organizzativo che svolga un ruolo attivo nella gestione della struttura del sistema-calcio.

Ogni confederazione calcistica è associata con uno ed un solo paese.
\newline
Come per le confederazioni, trasporremo su di un livello di astrazione superiore anche il concetto di paese.
Identificheremo come "paese" i concetti di regione, nazione, continente e persino il mondo intero.

Data la relazione che intercorre tra un paese ed una confederazione calcistica, queste ultime
potranno essere suddivise in tre grandi categorie:

\begin{itemize}
	\item Confederazioni nazionali;
	\item Confederazioni continentali;
	\item Confederazioni mondiali.
\end{itemize}

Sottolineiamo che tale suddivisione opera una certa semplificazione, in quanto
esistono confederazioni subcontinentali, confederazioni regionali e così via; ci è sembrato
superfluo elevare ulteriormente la granularità della nostra analisi.

Un'ulteriore aspetto concettuale che abbiamo rilevato in corso di ricerca è che -
come già accennato - le confederazioni sono organizzate in ordine gerarchico: le confederazioni
nazionali fanno parte di confederazioni continentali, che a loro volta sono
membri di confederazioni mondiali (es. la FIGC è membro della UEFA che a sua volta è membro
della FIFA).
Questo pattern si ripresenta per la quasi totalità delle confederazioni che abbiamo osservato:
le uniche e rare eccezioni fanno capo a confederazioni minori isolate, esterne alla FIFA,
che quindi non sono in associazione con nessun'altra confederazione.
Dalle ricerche effettuate abbiamo notato che a tutti gli effetti il sistema-calcio fa capo
alla FIFA e pertanto si è deciso di escludere il calcio esterno alla FIFA.


È indiscutibile che il concetto di confederazione calcistica, insieme a quello di paese,
formino la base sulla quale poggia tutta l'organizzazione calcistica mondiale.

\bigskip
\bigskip

Come detto in precedenza, ciascuna confederazione organizza diverse competizioni calcistiche.
Una competizione calcistica può essere analizzata secondo diversi criteri;
se consideriamo come elemento discriminante il tipo di squadra che può parteciparvi, è possibile
dividere le competizioni calcistiche in due categorie:
\begin{itemize}
	\item Competizioni per squadre di tipo club;
	\item Competizioni per squadre di tipo nazionale.
\end{itemize}

Tuttavia, se l'elemento discriminante è il format che caratterizza una competizione
calcistica, queste possono essere raggruppate in tre categorie:
\begin{itemize}
	\item Competizioni di tipo campionato;
	\item Competizioni di tipo torneo;
	\item Competizioni di tipo supercoppa.
\end{itemize}

In effetti, l'analisi delle competizioni calcistiche è ben più complessa di quella che stiamo descrivendo. Infatti, una competizione può svolgersi anche secondo una formula (girone all'italiana, eliminazione diretta, \dots),
che può variare in base all'edizione della competizione che si sta considerando,
e secondo cui può variare anche il numero di partecipanti. Risulta pertanto estremamente complicato astrarre coerentemente questo insieme di informazioni.

Nonostante le difficoltà, abbiamo cercato di conciliare e preservare entrambi i punti di vista sulle
competizioni calcistiche e, seguendo la traccia, abbiamo selezionato come preminente uno
dei due.

Nella traccia si fa un chiaro riferimento alle squadre di calcio, e pertanto ci è sembrato
più corretto prediligere la prospettiva che mettesse maggiormente in risalto questo aspetto;
pertanto, dati i nostri scopi, il concetto di competizione calcistica sarà in primo luogo
considerato in base al tipo di squadra che può parteciparvi, e solo in secondo luogo
in base al tipo di format.

Vale la pena fare un'importante precisazione: una confederazione nazionale non potrà organizzare competizioni
calcistiche per squadre di tipo nazionale.

\bigskip
\bigskip

Come ben noto, ogni competizione ha diverse edizioni, ciascuna delle quali si svolge entro
una specifica stagione calcistica.

Dalle ricerche effettuate è emerso che, nell'ambito di un'unica confederazione organizzativa e di uno specifico format, è possibile classificare le edizioni competitive per gradi di importanza (es. nella FIGC la serie A è il campionato di "primo
livello", la serie B il campionato di "secondo livello").

Abbiamo concluso che non può esistere una squadra di calcio che, nella
stessa stagione, partecipi a due diverse competizioni dello stesso format orgnizzate dalla stessa
confederazione.

Una stagione calcistica è a tutti gli effetti l'unità di misura temporale del calcio. E' scandita in un periodo che intercorre tra due anni consecutivi (es. Stagione 2023-2024).
Per le competizioni di squadre di tipo nazionale, il concetto di stagione calcistica
coincide con la durata di un anno solare (es. Stagione 2023-2023).

In una qualsiasi istanza di un'edizione competitiva, al termine delle competizioni previste, ad alcune squadre tra le partecipanti vengono assegnati dei trofei, in base alla loro posizione in classifica.

Un trofeo, quindi, risulta concettualmente associato a una specifica edizione di una competizione
calcistica; d'altra parte, le nostre ricerche hanno evidenziato la presenza di trofei calcistici
indipendenti dalle competizioni.


Pertanto, dal punto di vista astratto, si è reso necessario operare una distinzione tra i trofei calcistici
che saranno sempre correlati a delle competizioni, e premi calcistici che, invece
occorrono in maniera indipendente.

Un ultimo concetto chiave relativo al mondo del calcio è quello di "squadra di calcio".
Una squadra di calcio appartiene a una nazione e fa sempre parte della
confederazione della nazione a cui è associata.
Una squadra può far parte di due grandi categorie:
\begin{itemize}
	\item Squadra di calcio di tipo club;
	\item Squadra di calcio di tipo nazionale.
\end{itemize}

Dalle nostre ricerche è inoltre risultato che una squadra di calcio può partecipare
solo alle competizioni (del tipo di squadra corrispondente) organizzate dalla confederazione
di cui è membro o da confederazioni che contengono la confederazione calcistica di cui
è membro (es. la Salernitana, squadra di calcio di tipo club che appartiene alla FIGC,
può partecipare alle competizioni organizzate dalla FIGC, ma anche a quelle organizzate
dalla UEFA (Professore, la Salernitana in Europa è e rimarrà un sogno!),
ma non alle competizioni organizzate dalla COMBEBOL).

Un'ultima precisazione: abbiamo deciso di escludere dalla nostra trattazione sia il calcio
femminile che il calcio giovanile, non rappresentando questi un ampliamento efficace del presente progetto.
Aggiungeremo che, in ogni caso, sarebbe stato gestibile in modo essenzialmente identico, con la sola aggiunta di qualche attributo; è stato pertanto ritenuto uno sforzo inutile.

\bigskip
\bigskip

Questo è, in estrema sintesi, l'arricchimento che abbiamo deciso di conferire al nostro progetto.
Ricordiamo però che il principale scopo del nostro lavoro è costruire un sistema informativo
per la gestione di calciatori di tutto il mondo.

Le ricerche effettuate e le conclusioni a cui siamo giunti non rappresentano
un mero esercizio di stile, ma ci permetteranno di disporre di una descrizione più sofisticata dei singoli calciatori - a scapito della semplicità formale del progetto.

\bigskip
\bigskip

Ogni calciatore avrà una nazione di nascita, una o più nazionalità e,
come indicato da traccia, una carriera durante la quale può militare in diverse
squadre di calcio.

A questo proposito è doveroso fare una importante precisazione. Vista la natura più dettagliata
della nostra visione del progetto, anche il concetto di militanza di un calciatore in una squadra
di calcio sarà più dettagliato.

Il nostro concetto astratto di militanza sarà la presenza di un calciatore in una squadra
durante una stagione calcistica, visto che, come detto, la stagione è l'unità di misura
temporale del mondo calcio.

Abbiamo deciso quindi di non considerare la data di inizio e fine di una certa militanza in
quanto al nostro livello di dettaglio è impossibile da gestire in modo coerente e corretto,
mantenendo anche un certo livello di astrazione.
Infatti, a seguito delle nostre ricerche abbiamo notato che il regolamento vigente
prevede che un cambio di militanza di un calciatore da una squadra ad un'altra
possa avvenire solo in particolari periodi di tempo, detti "finestre di mercato",
che per altro sono variabili in base alla stagione considerata e alla confederazione
calcistica cui fanno riferimento; ecco il motivo per il quale non è possibile utilizzare
le date come richiesto da traccia.
Sarebbe, infatti, impossibile controllare correttamente le variazioni della militanza di un calciatore
durante la sua carriera e potrebbe accadere che il calciatore in questione in una stagione
possa militare in più squadre di quelle possibili secondo il regolamento.

Ultimo appunto riferito alla militanza: per le militanze di calciatori in squadre di tipo
nazionale abbiamo seguito le indicazioni del regolamento corrente, secondo il quale un calciatore
può giocare solo per una squadra nazionale in tutta la sua carriera. Tale squadra
deve rappresentare una nazione di cui il calciatore sia cittadino.

Abbiamo quindi gestito le militanze per stagione calcistica.

Durante una militanza in una squadra, un calciatore potrà giocare un certo numero di
partite di un'edizione di una competizione cui partecipa la squadra in cui milita.

Al gioco di un calciatore in un'edizione di una competizione calcistica saranno associate
alcune statistiche.

Dal punto di vista astratto, una statistica di gioco è un qualsiasi evento misurabile durante
una partita - e statisticamente rilevante - relativo al gioco del calcio
(numero di goal, numero di assist, \dots).

Visto lo scopo del nostro progetto, si è stabilito di uniformare le statistiche, definendo una
lista di statistiche predefinite generali ed altre specifiche per i calciatori che giocano
come portieri.

Similmente a quello che accade per una squadra, anche un calciatore potrà vincere dei trofei.
Come da traccia, tali trofei potranno essere individuali o di squadra.
Pertanto un trofeo potrà appartenere a due categorie:
\begin{itemize}
	\item Trofeo individuale;
	\item Trofeo di squadra.
\end{itemize}

Più specificamente, i trofei di squadra saranno vinti da una squadra e, per transitività, poi
assegnati a tutti i calciatori che militano nella suddetta squadra durante quella stagione.
Quelli individuali, invece, potranno essere vinti solo dai singoli calciatori.

Analogamente un calciatore potrà vincere premi calcistici individuali (es. Pallone d'oro).
Anche in questo caso, i premi potranno appartenere a due categorie:
\begin{itemize}
	\item Premio individuale;
	\item Premio di squadra.
\end{itemize}

Con l'importante differenza che un premio di squadra non potrà essere assegnato ad un
calciatore.

Un calciatore è inoltre caratterizzato da una posizione di gioco principale e possibilmente da una
serie di posizioni di gioco alternative che ne andranno a definire i ruoli.

Inoltre, come da traccia, ad un calciatore possono essere associati dei tag che mettono in
risalto le specialità del calciatore.

In aggiunta a ciò, prendendo ispirazione da FootballManager, si è deciso di includere nella
descrizione del calciatore anche una serie di attributi che ne descrivano in modo quantitativo
le capacità sotto diversi aspetti.

Infine, un calciatore potrà ritirarsi dal gioco in una certa data.

\bigskip
\bigskip

Come da traccia, tale DataBase sarà accessibile mediante un applicativo da diversi utenti.
In base al tipo di utente, e conseguentemente al suo livello di privilegio, saranno possibili
determinate azioni sul database.

Un utente semplice potrà solo visualizzare dati ed informazioni contenuti nel DataBase.
Un utente di tipo amministratore potrà anche modificare i dati e le informazioni relative a
squadre di calcio e calciatori (come richiesto da traccia).

Pertanto è stato scelto di rendere alcune informazioni permanenti, in modo tale da assicurare
che la struttura portante del DataBase, che rappresenta il sistema calcio, rimanga invariata.

È chiaro che poi il responsabile del DataBase potrà invece effettuare tutte le modifiche del caso.